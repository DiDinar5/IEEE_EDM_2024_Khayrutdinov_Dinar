\documentclass[10pt,conference,a4paper]{IEEEtran_EDM}
\IEEEoverridecommandlockouts
% The preceding line is only needed to identify funding in the first footnote. If that is unneeded, please comment it out.
\usepackage{cite}
\usepackage{amsmath,amssymb,amsfonts}
\usepackage{algorithmic}
\usepackage{graphicx}
\usepackage{textcomp}
\usepackage{xcolor}
\usepackage{listings}
\def\BibTeX{{\rm B\kern-.05em{\sc i\kern-.025em b}\kern-.08em
    T\kern-.1667em\lower.7ex\hbox{E}\kern-.125emX}}

\def\confheader{}

\makeatletter
\newcommand{\linebreakand}{%
  \end{@IEEEauthorhalign}
  \hfill\mbox{}\par
  \mbox{}\hfill\begin{@IEEEauthorhalign}
}
\makeatother

\usepackage{flushend} % <------------- used for automatically balance the columns on the last page. Be careful, read HOWTO XIV. LAST PAGE COLUMN EQUALIZATION 

\begin{document}

\markboth{\confheader}{}
\title{Data storage in a decentralized network}

\author{
\IEEEauthorblockN{line 1: 1\textsuperscript{st} Dinar Khayrutdinov}
\IEEEauthorblockA{\textit{line 2: Faculty of Automation and Computer Engineering}\\ 
\textit{(of Affiliation)} \\
\textit{line 3: Gubkin Russian State University of Oil and Gas}\\ 
\textit{(of Affiliation)}\\
line 4: Moscow, Russia \\
line 5: khayrutdinov.dd@gmail.com}
\and

\IEEEauthorblockN{line 1: 2\textsuperscript{nd} Denis Volkov}
\IEEEauthorblockA{\textit{line 2: Faculty of Automation and Computer Engineering}\\ 
\textit{(of Affiliation)} \\
\textit{line 3:Gubkin Russian State University of Oil and Gas}\\ 
\textit{(of Affiliation)}\\
line 4: Moscow, Russia\\
line 5: vda@asugubkin.ru }
}

\maketitle


\begin{abstract}
This article discusses a decentralized method of storing information in the educational field, as well as its advantages and disadvantages. Close attention is paid to security, accessibility and confidentiality. In addition to considering decentralized networks from the point of view of functionality, the article also describes the process of implementing distributed networks in Government Institutions such as a university, so that students, teachers and university administration have the opportunity to secure their data. The usual DBMS usage is compared with distributed file storage systems such as IPFS. The architecture of the decentralized application is also described in detail, this will facilitate direct user interaction with the file storage, as this application provides a convenient and intuitive interface. The link between the application and the blockchain is a smart contract that ensures the speed, transparency and reliability of the transaction. A decentralized application, smart contracts and an interface for information output have been developed. It is concluded that decentralized data storage is an innovation that has every chance of becoming an important component of Web3.

\end{abstract}

\begin{IEEEkeywords}
Decentralization, blockchain, smart contract, security, decentralized applications, distributed file systems, privacy, encryption.
\end{IEEEkeywords}

\section{Introduction}
% This document is a model and instructions for \LaTeX.
% Please observe the conference page limits. 
Today, information technology is developing at an exorbitant pace as never before. Human life is shrouded in all kinds of technology amenities that provide easy and fast interaction with the services offered. It is difficult to imagine a person who does not use social networks, a bank application and other innovations. In addition to the notable advantages of these technologies, there is another side that is associated with data leaks, hacks and hacker attacks. Information protection is one of the main goals of modern companies that face various problems when storing data on local devices. In this regard, the use of decentralized storage has become the most suitable option to meet the needs of businesses in data storage and management. If you follow the standards of the Russian Federation for ensuring information security in organizations [1], it becomes possible to secure your data, as well as maintain confidentiality. The document states that the organization that implements the technology must ensure security, confidentiality and accessibility for the user. Distributed file systems, blockchain, smart contracts and other decentralized technologies can provide the necessary functionality that will meet the criteria specified in the official document. 
The objectives of this work are the analysis of decentralized technologies, the development of necessary services, as well as practical guidance on the implementation of distributed file systems in an organization.


\section{Theory}

Decentralized storage is a way to store information on multiple nodes connected to a peer–to-peer network (P2P), for example, using BitTorrent or IPFS protocols.
In decentralized networks such as BitTorrent or IPFS, information is broken down into small parts and distributed across different network nodes. These nodes can be computers of any users connected to the network, which provides additional protection and fault tolerance [2]. By storing data on multiple nodes at the same time, even if one or more nodes fail or are attacked, the information remains accessible thanks to other nodes that have copies of the data.
Information uploaded to a storage system that does not have a central node is divided into small parts and distributed across several nodes. If the user wants to access the file, the network collects the separated parts from different nodes and combines them again into a file that can be downloaded. A more illustrative example is shown in Figure 1.

\begin{figure}[htbp]
\includegraphics[width=8.5cm, height=5cm]{fig1.png}
\caption{Getting data from IPFS.}
\label{fig}
\end{figure}

Also, nodes of a decentralized system do not have the ability to view or modify files, since the cryptographic hashing method automatically encrypts all data on the network. To gain access to them, users use their private keys, which protect information from third parties [3].
Using cloud storage provides significant advantages in data storage, including scalability, simplicity, cost savings, security and flexibility. Users can save their data to cloud storage in any remote location using a dedicated private network connection or the public Internet.

Consider the advantages of storing data in a decentralized network:
\begin{itemize}
\item Fault tolerance: When one or more nodes fail, data remains available thanks to other nodes in the network.
\item Security: The use of encryption and other methods ensures data protection in decentralized networks.
\item Efficiency: Node resources are used more efficiently, as the load is distributed among many participants.
\item Privacy: Users have more control over their data, as it is not stored in one central location that can be hacked or monitored.
\item Reliability: Decentralized systems can be more reliable, as they do not depend on a single node or organization.
Along with the advantages, there are disadvantages such as:
\item Access to information may take longer due to the dependence on the network of nodes.
\item Decentralized systems may be vulnerable to malicious nodes, which may pose a threat to data security.
\item Network infrastructure problems can also affect data availability.
\item The lack of standardization in decentralized systems makes it difficult to be compatible between different protocols.
\end{itemize}

Decentralized data storage is a new technology that has not yet been widely adopted, but may become an important element of Web3. Currently, users are looking for more convenient, efficient and secure ways to store information, so platforms with a decentralized structure, such as IPFS, are becoming more and more popular. Due to constant information leaks, rising storage costs and censorship in the traditional space, more and more people are thinking about using decentralized solutions. They can solve some of the problems of centralized analogues, but they also have their drawbacks. Currently, centralized data storage remains a popular choice for many people and will continue to occupy a significant market share, even with the growing demand for decentralized storage.
The complexity of the blockchain topic is due to complex algorithms, specific terms and other technical elements. There is a number of useful literature for those who are just starting their journey in the study of decentralized technologies, with the help of which the essence of the technology and the basic principles of the blockchain become clear [4]. In this regard, many organizations, Government Agencies and just ordinary users bypass this technology. For a more detailed immersion into the world of decentralization, it is proposed to introduce a discipline in educational institutions that will cover blockchain technologies from different angles and touch on more complex topics such as cryptography, blockchain platforms and decentralization in general [5]. Thanks to the training centers, there will be specialists who, in turn, will fill in the gaps of ignorance in society. 
Many people mistakenly compare the usual DBMS and IPFS. They do perform similar functions, the most important of which is data storage, but they are designed for different purposes [6]. As already mentioned, IPFS is a distributed file system, which means there is no central management. The DBMS, in turn, is designed for processing and managing structured data, usually in JSON format. Most DBMS are presented in the form of relational databases that store data in an organized table with columns and rows [7]. There are also problems with fault tolerance, if the database server broke down for a while, then all transactions that were transmitted at that time may not be saved, fortunately, backup mechanisms are now provided in case of threats to the server, but this does not negate the fact that the DBMS is subject to various failures. 
It happens that direct use of IPFS can make some actions difficult for a user who does not have confident knowledge of distributed networks, which can lead to confusion, as a result of which adverse consequences may occur. To do this, decentralized applications (DApp) were created, with their help, the user does not need to know how and where his data is stored, since a convenient and intuitive interface provides easy use of this platform [8]. In simple terms, a decentralized application serves as a shell between a human and a distributed file system. But in addition to the interface, DApp offers another stage of protection – it is identity verification through the blockchain [9]. The sequence of saving or viewing a file includes user identification, it is assumed that a hash key is stored in the blockchain, with which you can view the file stored in IPFS. The link in this chain is a smart contract. A smart contract is a code that automatically performs the actions prescribed inside the document [10]. A smart contract is valued for its immutability, transparency and reliability. An analogy can be given with a robot that performs its task according to its internal prescribed scenario. A popular programming language for writing smart contracts is Solidity, which runs on the Ethereum blockchain [11]. To fully understand the sequence from verification in a decentralized application to obtaining the desired file, it is proposed to consider in Figure 2.

\begin{figure}[htbp]
\includegraphics[width=8cm, height=7cm]{fig2.png}
\caption{The sequence of operation of decentralized technologies.}
\label{fig}
\end{figure}
The first thing a user needs to do is enter their Ethereum account and private key into a decentralized application. Next, with the help of a smart contract, the user is identified on the blockchain side. Upon successful verification, the hash key is returned, which is used to search for the file in IPFS. The final step is to output the desired file to the application interface. 
To implement this technology in a public Institution, it is proposed to follow the following steps [12]:
\begin{itemize}
\item Assessment of needs and opportunities
\item  Research and technology selection
\item Development of a legal and regulatory framework
\item Education and training
\item Integration and interaction with existing systems
\item System support
\end{itemize}

\section{Experimental results}
The implementation of a decentralized application that will store the received data in IPFS involves some steps and requires the use of certain tools and libraries. 

First of all, you need to create a DApp server part that will be responsible for receiving, reading and transferring files. Let's use Node.js based on JavaScript [13]. In addition to the basic functionality, you will need to turn to special libraries for working with web3 and IPFS. One of the most popular are Web3.js and js-IPFS. Listing 1 shows one of the main functions that accepts the hash key of a file, invokes a smart contract and stores the transaction in the blockchain.



\definecolor{codegreen}{rgb}{0,0.6,0}
\definecolor{codegray}{rgb}{0.5,0.5,0.5}
\definecolor{codepurple}{rgb}{0.58,0,0.82}
\definecolor{backcolour}{rgb}{0.95,0.95,0.92}

\lstdefinestyle{mystyle}{
    backgroundcolor=\color{backcolour},   
    commentstyle=\color{codegreen},
    keywordstyle=\color{magenta},
    numberstyle=\tiny\color{codegray},
    stringstyle=\color{codepurple},
    basicstyle=\ttfamily\footnotesize,
    breakatwhitespace=false,         
    breaklines=true,                 
    captionpos=b,                    
    keepspaces=true,                 
    numbers=left,                    
    numbersep=5pt,                  
    showspaces=false,                
    showstringspaces=false,
    showtabs=false,                  
    tabsize=2
}

\lstset{style=mystyle}


\begin{lstlisting}[language=JavaScript, caption=JavaScript code implementing saving to the blockchain]
import Web3 from 'web3';

async function addFileToBlockchain(fileHash) {
    const web3 = new Web3(Web3.givenProvider || "http://localhost:8545");
    const contractABI = [/* smart contract ABI */];
    const contractAddress = "/* smart contract address */";
    const contract = new web3.eth.Contract(contractABI, contractAddress);

    const accounts = await web3.eth.getAccounts();
    const receipt = await contract.methods.addFile(fileHash).send({ from: accounts[0] });
    return receipt;
}
\end{lstlisting}



After the server part is created, it is proposed to create a smart contract that will link the server written earlier and the blockchain. It is designed to manage and store user hashes. As shown in listing 2, a so-called dictionary is created, which stores data on the principle of key and value. The key is the user's address, and the hash value of the file that is associated with a specific user. In addition to the dictionary, two functions are created, one of which is responsible for adding the transmitted hash file to the user's hash array, and the other provides saved hashes of files associated with the specified user address.



\lstdefinelanguage{Solidity}{
    keywords={typeof, new, true, false, catch, function, return, null, catch, switch, var, if, in, while, do, else, case, break, contract, public, private, view, returns, memory, mapping, address, string, push},
    keywordstyle=\color{blue}\bfseries,
    ndkeywords={class, export, boolean, throw, implements, import, this},
    ndkeywordstyle=\color{darkgray}\bfseries,
    identifierstyle=\color{black},
    sensitive=false,
    comment=[l]{//},
    morecomment=[s]{/*}{*/},
    commentstyle=\color{purple}\ttfamily,
    stringstyle=\color{red}\ttfamily,
    morestring=[b]',
    morestring=[b]"
}

\lstset{
   language=Solidity,
   backgroundcolor=\color{white},
   extendedchars=true,
   basicstyle=\footnotesize\ttfamily,
   showstringspaces=false,
   showspaces=false,
   numbers=left,
   numberstyle=\footnotesize,
   numbersep=9pt,
   tabsize=2,
   breaklines=true,
   showtabs=false,
   captionpos=b
}


\begin{lstlisting}[caption=Solidity code for a smart contract]
// SPDX-License-Identifier: MIT
pragma solidity ^0.8.0;

contract FileStorage {
    mapping(address => string[]) private userFiles;

    function addFile(string memory fileHash) public {
        userFiles[msg.sender].push(fileHash);
    }

    function getFiles() public view returns (string[] memory) {
        return userFiles[msg.sender];
    }
}
\end{lstlisting}

The React framework is responsible for the interface [14]. This is one of the most popular libraries for the JavaScript programming language, it creates dynamic interfaces, besides it does not require much knowledge in development. The task of the interface is to output information with a valid request, if the server side has caught an error, then the error information should be displayed on the screen. Such situations are not uncommon, exceptions can be either due to a non-existent file or user, or due to breakdowns on the server. The main thing is to handle these errors correctly so that the program does not crash, for this it is necessary to anticipate which exceptions may occur in a given situation.

Thorough testing of all aspects of the program is the key to the stability of the entire system. There are several testing options, for example, for testing and deploying smart contracts, it is proposed to use the Truffle framework, this library has integration with the local Ganache blockchain environment. Simply put, this is a personal local blockchain for testing, with which you can track the transaction of a smart contract. When the server starts, the transaction is saved in a block. Figure 3 shows the Ganache interface, which displays the user's saved address and balance.



\begin{figure}[htbp]
\includegraphics[width=8cm, height=3cm]{fig3.png}
\caption{The interface of the local Ganache blockchain.}
\label{fig}
\end{figure}


\section{Conclusions }
In this paper, a decentralized method of storing information in the educational field is considered. The process of implementing distributed networks in Government Institutions is described. DBMS and their differences from IPFS are also considered in detail. Dap, smart contract, user interface have been developed and all interactions with third-party frameworks have been configured. The application has been tested and has no internal errors.

\begin{thebibliography}{00}
\bibitem{b1} G. Eason, B. Noble, and I. N. Sneddon, ``On certain integrals of Lipschitz-Hankel type involving products of Bessel functions,'' Phil. Trans. Roy. Soc. London, vol. A247, pp. 529--551, April 1955. \textit{(references)}
\bibitem{b2} J. Clerk Maxwell, A Treatise on Electricity and Magnetism, 3rd ed., vol. 2. Oxford: Clarendon, 1892, pp.68--73.
\bibitem{b3} I. S. Jacobs and C. P. Bean, ``Fine particles, thin films and exchange anisotropy,'' in Magnetism, vol. III, G. T. Rado and H. Suhl, Eds. New York: Academic, 1963, pp. 271--350.
\bibitem{b4} K. Elissa, ``Title of paper if known,'' unpublished.
\bibitem{b5} R. Nicole, ``Title of paper with only first word capitalized,'' J. Name Stand. Abbrev., in press.
\bibitem{b6} Y. Yorozu, M. Hirano, K. Oka, and Y. Tagawa, ``Electron spectroscopy studies on magneto-optical media and plastic substrate interface,'' IEEE Transl. J. Magn. Japan, vol. 2, pp. 740--741, August 1987 [Digests 9th Annual Conf. Magnetics Japan, p. 301, 1982].
\bibitem{b7} M. Young, The Technical Writer's Handbook. Mill Valley, CA: University Science, 1989.
\end{thebibliography}
\vspace{12pt}
\color{red}

\end{document}
